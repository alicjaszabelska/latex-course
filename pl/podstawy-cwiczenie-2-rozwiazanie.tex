\documentclass{article}
\usepackage[polish]{babel}
\usepackage[utf8]{inputenc}
\usepackage[T1]{fontenc}

\usepackage{amsmath}
\begin{document}
Niech $X_1, X_2, \ldots, X_n$ będzie ciągiem niezależnych
zmiennych losowych o jednakowych rozkładach z parametrami 
$\operatorname{E}[X_i] = \mu$ i $\operatorname{Var}[X_i] = \sigma^2 < \infty$, 
oraz niech
\begin{equation*}
S_n = \frac{1}{n}\sum_{i=1}^{n} X_i
\end{equation*}
oznaczyć ich średnią. Następnie, gdy $n$ zbliża się do 
nieskończoności, wtedy zmienne losowe $\sqrt{n}(S_n - \mu)$ 
zbiegają do rozkładu normalnego $N(0, \sigma^2)$.

% dodatkowe punkty: N oznaczające normalność jest zwykle kaligrafowaną
% czcionką; możesz to uzyskać za pomocą $\mathcal{N}(0, \sigma^2)$.

\end{document}
