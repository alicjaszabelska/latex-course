\documentclass{beamer}

\input{preamble.tex}

\subtitle{Cześć 1: Podstawy}

\begin{document}

%%%%%%%%%%%%%%%%%%%%%%%%%%%%%%%%%%%%%%%%%%%%%%%%%%%%%%%%%%%%%%%%%%%%%%%%%%%%%%%
%%%%%%%%%%%%%%%%%%%%%%%%%%%%%%%%%%%%%%%%%%%%%%%%%%%%%%%%%%%%%%%%%%%%%%%%%%%%%%%
%%%%%%%%%%%%%%%%%%%%%%%%%%%%%%%%%%%%%%%%%%%%%%%%%%%%%%%%%%%%%%%%%%%%%%%%%%%%%%%
\begin{frame}{}
\titlepage
\end{frame}
%%%%%%%%%%%%%%%%%%%%%%%%%%%%%%%%%%%%%%%%%%%%%%%%%%%%%%%%%%%%%%%%%%%%%%%%%%%%%%%
%%%%%%%%%%%%%%%%%%%%%%%%%%%%%%%%%%%%%%%%%%%%%%%%%%%%%%%%%%%%%%%%%%%%%%%%%%%%%%%
%%%%%%%%%%%%%%%%%%%%%%%%%%%%%%%%%%%%%%%%%%%%%%%%%%%%%%%%%%%%%%%%%%%%%%%%%%%%%%%
\begin{frame}{Dlaczego \LaTeX{}?}
\begin{itemize}
\item Tworzy piękne dokumenty
\begin{itemize}
\item Szczególnie te, uwzględniające matematykę
\end{itemize}
%
\item Został stworzony przez naukowców, dla naukowców
\begin{itemize}
\item Istnieje duża i aktywna społeczność
\end{itemize}
%
\item Jest potężnym narzędziem --- możesz je rozwijać
\begin{itemize}
\item Pakiety do artykułów, prezentacji, arkuszy kalkulacyjnych, \ldots
\end{itemize}
\end{itemize}
\end{frame}

%%%%%%%%%%%%%%%%%%%%%%%%%%%%%%%%%%%%%%%%%%%%%%%%%%%%%%%%%%%%%%%%%%%%%%%%%%%%%%%
%%%%%%%%%%%%%%%%%%%%%%%%%%%%%%%%%%%%%%%%%%%%%%%%%%%%%%%%%%%%%%%%%%%%%%%%%%%%%%%
%%%%%%%%%%%%%%%%%%%%%%%%%%%%%%%%%%%%%%%%%%%%%%%%%%%%%%%%%%%%%%%%%%%%%%%%%%%%%%%
\begin{frame}[fragile]{Jak działa?}
\begin{itemize}
\item Tworzysz swój dokument \texttt{samym tekstem} z użyciem \cmd{komend}, które
opisują ich strukturę i znaczenie.
\item Program \texttt{latex} przetwarza tekst oraz komendy, aby wyprodukować
pięknie sformatowany dokument.
\end{itemize}
\vskip 2ex
\begin{center}
\begin{minted}[frame=single]{latex}
Deszcz w Hiszpanii pada \emph{głównie} na równinie.
\end{minted}
\vskip 2ex
\tikz\node[single arrow,fill=gray,font=\ttfamily\bfseries,%
  rotate=270,xshift=-1em]{latex};
\vskip 2ex
\fbox{Deszcz w Hiszpanii pada \emph{głównie} na równinie.}
\end{center}
\end{frame}

%%%%%%%%%%%%%%%%%%%%%%%%%%%%%%%%%%%%%%%%%%%%%%%%%%%%%%%%%%%%%%%%%%%%%%%%%%%%%%%
%%%%%%%%%%%%%%%%%%%%%%%%%%%%%%%%%%%%%%%%%%%%%%%%%%%%%%%%%%%%%%%%%%%%%%%%%%%%%%%
%%%%%%%%%%%%%%%%%%%%%%%%%%%%%%%%%%%%%%%%%%%%%%%%%%%%%%%%%%%%%%%%%%%%%%%%%%%%%%%
\begin{frame}[fragile]{Więcej przykładów komend i efektu ich działania\ldots}
\begin{exampletwoup}
\begin{itemize}
\item Herbata
\item Mleko
\item Ciasteczka
\end{itemize}
\end{exampletwoup}
\vskip 2ex
\begin{exampletwoup}
\begin{figure}
\includegraphics{gerbil}
\end{figure}
\end{exampletwoup}
\vskip 2ex
\begin{exampletwoup}
\begin{equation}
\alpha + \beta + 1
\end{equation}
\end{exampletwoup}

\tiny{Image license: \href{https://pixabay.com/en/animal-apple-attractive-beautiful-1239390/}{CC0}}
\end{frame}

%%%%%%%%%%%%%%%%%%%%%%%%%%%%%%%%%%%%%%%%%%%%%%%%%%%%%%%%%%%%%%%%%%%%%%%%%%%%%%%
%%%%%%%%%%%%%%%%%%%%%%%%%%%%%%%%%%%%%%%%%%%%%%%%%%%%%%%%%%%%%%%%%%%%%%%%%%%%%%%
%%%%%%%%%%%%%%%%%%%%%%%%%%%%%%%%%%%%%%%%%%%%%%%%%%%%%%%%%%%%%%%%%%%%%%%%%%%%%%%
\begin{frame}[fragile]{Zmiana nastawienia}

\begin{itemize}
\item Używaj komend, aby opisać `co to jest', a nie `jak to wygląda'.
\item Skup się na zawartości.
\item Pozwól \LaTeX{}'owi wykonać swoje zadanie.
\end{itemize}
\end{frame}

%%%%%%%%%%%%%%%%%%%%%%%%%%%%%%%%%%%%%%%%%%%%%%%%%%%%%%%%%%%%%%%%%%%%%%%%%%%%%%%
%%%%%%%%%%%%%%%%%%%%%%%%%%%%%%%%%%%%%%%%%%%%%%%%%%%%%%%%%%%%%%%%%%%%%%%%%%%%%%%
%%%%%%%%%%%%%%%%%%%%%%%%%%%%%%%%%%%%%%%%%%%%%%%%%%%%%%%%%%%%%%%%%%%%%%%%%%%%%%%
\section{Podstawy}

%%%%%%%%%%%%%%%%%%%%%%%%%%%%%%%%%%%%%%%%%%%%%%%%%%%%%%%%%%%%%%%%%%%%%%%%%%%%%%%
%%%%%%%%%%%%%%%%%%%%%%%%%%%%%%%%%%%%%%%%%%%%%%%%%%%%%%%%%%%%%%%%%%%%%%%%%%%%%%%
%%%%%%%%%%%%%%%%%%%%%%%%%%%%%%%%%%%%%%%%%%%%%%%%%%%%%%%%%%%%%%%%%%%%%%%%%%%%%%%
\subsection{Zacznijmy}
\begin{frame}[fragile]{\insertsubsection}
\begin{itemize}
\item Minimalny dokument \LaTeX{}:
\inputminted[frame=single]{latex}{basics.tex}
\item Komendy zaczynają się od przycisku \emph{backslash} \keystrokebftt{\bs}.
\item Każdy dokument zaczyna się komendą \cmdbs{documentclass}.
\item \emph{Argument} w nawiasach klamrowych \keystrokebftt{\{} \keystrokebftt{\}} mówi \LaTeX{}'owi jaki rodzaj dokumentu tworzymy: \bftt{article}.
\item Znak procentu \keystrokebftt{\%} rozpoczyna \emph{komentarz} --- \LaTeX{}
zignoruje resztę linii przy kompilacji.
\end{itemize}
\end{frame}

%%%%%%%%%%%%%%%%%%%%%%%%%%%%%%%%%%%%%%%%%%%%%%%%%%%%%%%%%%%%%%%%%%%%%%%%%%%%%%%
%%%%%%%%%%%%%%%%%%%%%%%%%%%%%%%%%%%%%%%%%%%%%%%%%%%%%%%%%%%%%%%%%%%%%%%%%%%%%%%
%%%%%%%%%%%%%%%%%%%%%%%%%%%%%%%%%%%%%%%%%%%%%%%%%%%%%%%%%%%%%%%%%%%%%%%%%%%%%%%
\begin{frame}[fragile]{\insertsubsection{} z \wllogo}
\begin{itemize}
\item Overleaf jest platformą internetową do pisania dokumentów w \LaTeX'u.
\item `Kompiluje' plik \LaTeX{} automatycznie, aby pokazać efekt końcowy.
\vskip 2em
\begin{center}
\fbox{\href{\wlnewdoc{podstawy.tex}}{%
Naciśnij tutaj, aby otworzyć przykładowy dokument w \wllogo{}}}
\\[1ex]\scriptsize{}
Dla najlepszych resultatów, proszę użyj \href{http://www.google.com/chrome}{Google Chrome} lub aktualnej wersji \href{http://www.mozilla.org/en-US/firefox/new/}{FireFox}.
\end{center}
\vskip 2ex
\item Podczas przeglądania kolejnych slajdów wypróbuj przykłady, wpisując je
do przykładowego dokumentu na Overleaf.
\item \textbf{Naprawdę, powinieneś je wypróbować w trakcie!}
\end{itemize}
\end{frame}

%%%%%%%%%%%%%%%%%%%%%%%%%%%%%%%%%%%%%%%%%%%%%%%%%%%%%%%%%%%%%%%%%%%%%%%%%%%%%%%
%%%%%%%%%%%%%%%%%%%%%%%%%%%%%%%%%%%%%%%%%%%%%%%%%%%%%%%%%%%%%%%%%%%%%%%%%%%%%%%
%%%%%%%%%%%%%%%%%%%%%%%%%%%%%%%%%%%%%%%%%%%%%%%%%%%%%%%%%%%%%%%%%%%%%%%%%%%%%%%
\subsection{Skład tekstu}
\begin{frame}[fragile]{\insertsubsection{}}
\small
\begin{itemize}
\item Napisz swój tekst pomiędzy \cmdbegin{document} oraz \cmdend{document}.
\item W większości przypadków możesz po prostu normalnie wpisać tekst.
\begin{exampletwouptiny}
Słowa są oddzielone jedną lub więcej
spacjami.

Akapity są oddzielone jedną lub więcej 
pustymi liniami.
\end{exampletwouptiny}
\item Spacje w pliku źródłowym są redukowane w pliku wynikowym.
\begin{exampletwouptiny}
Deszcz    w    Hiszpanii     pada 
głównie        na       równinie.
\end{exampletwouptiny}
\end{itemize}
\end{frame}

%%%%%%%%%%%%%%%%%%%%%%%%%%%%%%%%%%%%%%%%%%%%%%%%%%%%%%%%%%%%%%%%%%%%%%%%%%%%%%%
%%%%%%%%%%%%%%%%%%%%%%%%%%%%%%%%%%%%%%%%%%%%%%%%%%%%%%%%%%%%%%%%%%%%%%%%%%%%%%%
%%%%%%%%%%%%%%%%%%%%%%%%%%%%%%%%%%%%%%%%%%%%%%%%%%%%%%%%%%%%%%%%%%%%%%%%%%%%%%%
\begin{frame}[fragile]{\insertsubsection{}: Uwagi}
\small
\begin{itemize}
\item Cudzysłowy są nieco skomplikowane:\\
użyj przycisku grawisa \keystroke{\`{}} po lewej stronie oraz przycisk apostrofu \keystroke{\'{}} po prawej.
\begin{exampletwouptiny}
Pojedynczy cudzysłów: `tekst'.

Podwójny cudzysłów: ``tekst''.
\end{exampletwouptiny}

\item Niektóre popularne znaki mają specjalne znaczenie w \LaTeX u:\\[1ex]
\begin{tabular}{cl}
\keystrokebftt{\%} & znak procentu              \\
\keystrokebftt{\#} & znak krzyżyka (hash)       \\
\keystrokebftt{\&} & znak et (oraz)             \\
\keystrokebftt{\$} & znak dolara                \\
\end{tabular}
\item Jeśli po prostu je wpiszesz, pojawi się błąd. Jeśli chcesz, aby pojawiły się w
końcowym dokumencie, musisz je poprzedzić ukośnikiem wstecznym (backslash).
\begin{exampletwoup}
\$\%\&\#!
\end{exampletwoup}
\end{itemize}
\end{frame}

%%%%%%%%%%%%%%%%%%%%%%%%%%%%%%%%%%%%%%%%%%%%%%%%%%%%%%%%%%%%%%%%%%%%%%%%%%%%%%%
%%%%%%%%%%%%%%%%%%%%%%%%%%%%%%%%%%%%%%%%%%%%%%%%%%%%%%%%%%%%%%%%%%%%%%%%%%%%%%%
%%%%%%%%%%%%%%%%%%%%%%%%%%%%%%%%%%%%%%%%%%%%%%%%%%%%%%%%%%%%%%%%%%%%%%%%%%%%%%%
\begin{frame}[fragile]{Radzenie sobie z błędami - \textit{Errare humanum est}}
\begin{itemize}
\item \LaTeX{} może się pogubić, gdy próbuje skompilować twój dokument. Jeśli to
robi, zatrzymuje się z błędem, który musisz naprawić, zanim program mógł 
wygenerować plik wynikowy.
\item Na przykład, jeśli błędnie zapiszesz \cmdbs{emph} jako \cmdbs{meph}, \LaTeX{}
zatrzyma się z błędem ``undefined control sequence'', ponieważ ``meph'' nie jest
jedną z komend znanych programowi.
\end{itemize}
\begin{block}{Porady dotyczące błędów}
\begin{enumerate}
\item Nie panikuj! Błędy się zdarzają.
\item Napraw je, gdy tylko się pojawią --- jeśli to, co właśnie wpisałeś, spowodowało błąd,
możesz tam rozpocząć debugowanie.
\item Jeśli jest wiele błędów, zacznij od pierwszego --- przyczyna może
być nawet wcześniej.
\end{enumerate}
\end{block}
\end{frame}

%%%%%%%%%%%%%%%%%%%%%%%%%%%%%%%%%%%%%%%%%%%%%%%%%%%%%%%%%%%%%%%%%%%%%%%%%%%%%%%
%%%%%%%%%%%%%%%%%%%%%%%%%%%%%%%%%%%%%%%%%%%%%%%%%%%%%%%%%%%%%%%%%%%%%%%%%%%%%%%
%%%%%%%%%%%%%%%%%%%%%%%%%%%%%%%%%%%%%%%%%%%%%%%%%%%%%%%%%%%%%%%%%%%%%%%%%%%%%%%
\begin{frame}[fragile]{Skład tekstu Ćwiczenie 1}

\begin{block}{Wpisz poniższy tekst w \LaTeX u:
\footnote{\url{http://en.wikipedia.org/wiki/Economy_of_the_United_States}}}
In March 2006, Congress raised that ceiling an additional \$0.79 trillion to
\$8.97 trillion, which is approximately 68\% of GDP. As of October 4, 2008, the
``Emergency Economic Stabilization Act of 2008'' raised the current debt ceiling
to \$11.3 trillion.
\end{block}
\vskip 2ex
\begin{center}
\fbox{\href{\wlnewdoc{podstawy-cwiczenie-1.tex}}{%
Kliknij, aby otworzyć to ćwiczenie w \wllogo{}}}
\end{center}

\begin{itemize}
\item Wskazówka: uważaj na znaki o specjalnym znaczeniu!
\item Kiedy już spróbujesz,
\fbox{\href{\wlnewdoc{podstawy-cwiczenie-1-rozwiazanie.tex}}{%
kliknij, aby zobaczyć rozwiązanie}}.
\end{itemize}
\end{frame}

%%%%%%%%%%%%%%%%%%%%%%%%%%%%%%%%%%%%%%%%%%%%%%%%%%%%%%%%%%%%%%%%%%%%%%%%%%%%%%%
%%%%%%%%%%%%%%%%%%%%%%%%%%%%%%%%%%%%%%%%%%%%%%%%%%%%%%%%%%%%%%%%%%%%%%%%%%%%%%%
%%%%%%%%%%%%%%%%%%%%%%%%%%%%%%%%%%%%%%%%%%%%%%%%%%%%%%%%%%%%%%%%%%%%%%%%%%%%%%%
\subsection{Skład tekstu matematycznego}
\begin{frame}[fragile]{\insertsubsection{}: symbole dolara}
\begin{itemize}
\item Dlaczego znaki dolara \keystrokebftt{\$} specjalne? Używamy ich do zaznaczania
matematycznych wyrażeń w tekście.\\[1ex]
\begin{exampletwouptiny}
% niezbyt dobrze:
Niech a i b będą dwiema różnymi 
liczbami naturalnymi, oraz niech c = a - b + 1.

% much better:
Niech $a$ i $b$ będą dwiema różnymi 
liczbami naturalnymi, oraz niech $c = a - b + 1$.
\end{exampletwouptiny}
\item Zawsze używaj znaków dolara w parach - jeden na początku wyrażenia, a drugi
na końcu.
\item \LaTeX{} automatycznie obsługuje odstępy; ignoruje twoje spacje.
\begin{exampletwouptiny}
Niech $y=mx+b$ będzie \ldots

Niech $y = m x + b$ będzie \ldots
\end{exampletwouptiny}
\end{itemize}
\end{frame}

%%%%%%%%%%%%%%%%%%%%%%%%%%%%%%%%%%%%%%%%%%%%%%%%%%%%%%%%%%%%%%%%%%%%%%%%%%%%%%%
%%%%%%%%%%%%%%%%%%%%%%%%%%%%%%%%%%%%%%%%%%%%%%%%%%%%%%%%%%%%%%%%%%%%%%%%%%%%%%%
%%%%%%%%%%%%%%%%%%%%%%%%%%%%%%%%%%%%%%%%%%%%%%%%%%%%%%%%%%%%%%%%%%%%%%%%%%%%%%%
\begin{frame}[fragile]{\insertsubsection{}: Notacja}
\begin{itemize}
\item Użyj znaku dzióbka \keystrokebftt{\^} dla indeksów górnych, a podkreślenia \keystrokebftt{\_} dla indeksów dolnych.
\begin{exampletwouptiny}
$y = c_2 x^2 + c_1 x + c_0$
\end{exampletwouptiny}
\vskip 2ex

\item Użyj nawiasów klamrowych\keystrokebftt{\{} \keystrokebftt{\}}, aby zgrupować
indeksy górne i dolne.
\begin{exampletwouptiny}
$F_n = F_n-1 + F_n-2$     % ups!

$F_n = F_{n-1} + F_{n-2}$ % ok!
\end{exampletwouptiny}
\vskip 2ex

\item Istnieją polecenia dotyczące liter greckich i typowych oznaczeń.
\begin{exampletwouptiny}
$\mu = A e^{Q/RT}$

$\Omega = \sum_{k=1}^{n} \omega_k$
\end{exampletwouptiny}
\end{itemize}
\end{frame}

%%%%%%%%%%%%%%%%%%%%%%%%%%%%%%%%%%%%%%%%%%%%%%%%%%%%%%%%%%%%%%%%%%%%%%%%%%%%%%%
%%%%%%%%%%%%%%%%%%%%%%%%%%%%%%%%%%%%%%%%%%%%%%%%%%%%%%%%%%%%%%%%%%%%%%%%%%%%%%%
%%%%%%%%%%%%%%%%%%%%%%%%%%%%%%%%%%%%%%%%%%%%%%%%%%%%%%%%%%%%%%%%%%%%%%%%%%%%%%%
\begin{frame}[fragile]{\insertsubsection{}: Wyświetlanie równań}
\begin{itemize}
\item Jeśli jest duży i przerażający, \emph{wyświetl} go w osobnej linii, używając
\cmdbegin{równanie} i \cmdend{równanie}.\\[2ex]
\begin{exampletwouptiny}
Pierwiastki równania kwadratowego
są podane przez
\begin{equation}
x = \frac{-b \pm \sqrt{b^2 - 4ac}}
         {2a}
\end{equation}
gdzie $a$, $b$ i $c$ są \ldots
\end{exampletwouptiny}
\vskip 1em
{\scriptsize Uwaga: \LaTeX{} przeważnie ignoruje spacje w wyrażeniach, ale nie 
radzi sobie z pustymi wierszami w równaniach --- nie wstawiaj tam pustych wierszy.}
\end{itemize}
\end{frame}

%%%%%%%%%%%%%%%%%%%%%%%%%%%%%%%%%%%%%%%%%%%%%%%%%%%%%%%%%%%%%%%%%%%%%%%%%%%%%%%
%%%%%%%%%%%%%%%%%%%%%%%%%%%%%%%%%%%%%%%%%%%%%%%%%%%%%%%%%%%%%%%%%%%%%%%%%%%%%%%
%%%%%%%%%%%%%%%%%%%%%%%%%%%%%%%%%%%%%%%%%%%%%%%%%%%%%%%%%%%%%%%%%%%%%%%%%%%%%%%
\begin{frame}[fragile]{Interludium: Środowiska}
\begin{itemize}
\item \bftt{equation} jest \emph{środowiskiem} --- kontekstem.
\item Komenda może generować różne wyniki w różnych kontekstach.
\begin{exampletwouptiny}
Możemy napisać
$ \Omega = \sum_{k=1}^{n} \omega_k $
w tekście, lub możemy napisać 
\begin{equation}
  \Omega = \sum_{k=1}^{n} \omega_k
\end{equation}
aby go wyświetlić w oddzielnej linii.
\end{exampletwouptiny}
\vskip 2ex
\item Zwróć uwagę, że $\Sigma$ jest większa w środowisku \bftt{equation} i jak indeksy dolne i górne zmieniają położenie, mimo że użyliśmy tych samych poleceń.
\vskip 1em
{\scriptsize Właściwie moglibyśmy napisać \bftt{\$...\$} jako
\cmdbegin{math}\bftt{...}\cmdend{math}.}
\end{itemize}
\end{frame}

%%%%%%%%%%%%%%%%%%%%%%%%%%%%%%%%%%%%%%%%%%%%%%%%%%%%%%%%%%%%%%%%%%%%%%%%%%%%%%%
%%%%%%%%%%%%%%%%%%%%%%%%%%%%%%%%%%%%%%%%%%%%%%%%%%%%%%%%%%%%%%%%%%%%%%%%%%%%%%%
%%%%%%%%%%%%%%%%%%%%%%%%%%%%%%%%%%%%%%%%%%%%%%%%%%%%%%%%%%%%%%%%%%%%%%%%%%%%%%%
\begin{frame}[fragile]{Interludium: Środowiska}
\begin{itemize}
\item Komendy \cmdbs{begin} i \cmdbs{end} są używane do tworzenia wielu różnych
środowisk.
\vskip 2ex

\item Środowiska \bftt{itemize} i \bftt{enumerate} tworzą listy.
\begin{exampletwouptiny}
\begin{itemize} % for bullet points
\item Ciasteczka
\item Herbata
\end{itemize}

\begin{enumerate} % for numbers
\item Ciasteczka
\item Herbata
\end{enumerate}
\end{exampletwouptiny}
\end{itemize}
\end{frame}

%%%%%%%%%%%%%%%%%%%%%%%%%%%%%%%%%%%%%%%%%%%%%%%%%%%%%%%%%%%%%%%%%%%%%%%%%%%%%%%
%%%%%%%%%%%%%%%%%%%%%%%%%%%%%%%%%%%%%%%%%%%%%%%%%%%%%%%%%%%%%%%%%%%%%%%%%%%%%%%
%%%%%%%%%%%%%%%%%%%%%%%%%%%%%%%%%%%%%%%%%%%%%%%%%%%%%%%%%%%%%%%%%%%%%%%%%%%%%%%
\begin{frame}[fragile]{Interludium: Pakiety}

\begin{itemize}
\item Wszystkie polecenia i środowiska, z których korzystaliśmy do tej pory, są wbudowane
w \LaTeX u.

\item \emph{Pakiety} to biblioteki dodatkowych komend i środowisk. 
Istnieją tysiące darmowo dostępnych pakietów.

\item Musimy załadować każdy z pakietów, których chcemy użyć, za pomocą komendy
\cmdbs{usepackage} w \emph{preambule}.

\item Przykład: pakiet \bftt{amsmath} z Amerykańskiego Towarzystwa Matematycznego.
\begin{minted}[fontsize=\small,frame=single]{latex}
\documentclass{article}
\usepackage{amsmath} % preambuła
\begin{document}
% teraz możemy używać poleceń z pakietu amsmath ...
\end{document}
\end{minted}
\end{itemize}
\end{frame}

%%%%%%%%%%%%%%%%%%%%%%%%%%%%%%%%%%%%%%%%%%%%%%%%%%%%%%%%%%%%%%%%%%%%%%%%%%%%%%%
%%%%%%%%%%%%%%%%%%%%%%%%%%%%%%%%%%%%%%%%%%%%%%%%%%%%%%%%%%%%%%%%%%%%%%%%%%%%%%%
%%%%%%%%%%%%%%%%%%%%%%%%%%%%%%%%%%%%%%%%%%%%%%%%%%%%%%%%%%%%%%%%%%%%%%%%%%%%%%%
\begin{frame}[fragile]{\insertsubsection{}: Przykłady z \bftt{amsmath}}
\begin{itemize}
\item Użyj \bftt{equation*} (``equation-star'') dla równań nienumerowanych.
\begin{exampletwouptiny}
\begin{equation*}
  \Omega = \sum_{k=1}^{n} \omega_k
\end{equation*}
\end{exampletwouptiny}
\item \LaTeX{} traktuje sąsiednie litery jako zmienne pomnożone przez siebie, co nie zawsze jest pożądane. \bftt{amsmath} definiuje komendy dla wielu powszechnych operatorów matematycznych.
\begin{exampletwouptiny}
\begin{equation*} % źle!
 min_{x,y} (1-x)^2 + 100(y-x^2)^2
\end{equation*}
\begin{equation*} % dobrze!
\min_{x,y}{(1-x)^2 + 100(y-x^2)^2}
\end{equation*}
\end{exampletwouptiny}
\item Możesz użyć \cmdbs{operatorname} dla innych.
\begin{exampletwouptiny}
\begin{equation*}
\beta_i =
\frac{\operatorname{Cov}(R_i, R_m)}
     {\operatorname{Var}(R_m)}
\end{equation*}
\end{exampletwouptiny}
\end{itemize}
\end{frame}

%%%%%%%%%%%%%%%%%%%%%%%%%%%%%%%%%%%%%%%%%%%%%%%%%%%%%%%%%%%%%%%%%%%%%%%%%%%%%%%
%%%%%%%%%%%%%%%%%%%%%%%%%%%%%%%%%%%%%%%%%%%%%%%%%%%%%%%%%%%%%%%%%%%%%%%%%%%%%%%
%%%%%%%%%%%%%%%%%%%%%%%%%%%%%%%%%%%%%%%%%%%%%%%%%%%%%%%%%%%%%%%%%%%%%%%%%%%%%%%
\begin{frame}[fragile]{\insertsubsection{}: Przykłady z \bftt{amsmath}}
\begin{itemize}{\small
\item Dopasuj ciąg równań do znaku równości
\begin{align*}
(x+1)^3 &= (x+1)(x+1)(x+1) \\
        &= (x+1)(x^2 + 2x + 1) \\
        &= x^3 + 3x^2 + 3x + 1
\end{align*}
ze środowiskiem \bftt{align*}.

% for whatever reason, this doesn't play well with the twoup environment
\begin{minted}[fontsize=\small,frame=single]{latex}
\begin{align*}
(x+1)^3 &= (x+1)(x+1)(x+1) \\
        &= (x+1)(x^2 + 2x + 1) \\
        &= x^3 + 3x^2 + 3x + 1
\end{align*}
\end{minted}
\item Znak \keystrokebftt{\&} oddziela lewą kolumnę (przed znakiem
$=$) od prawej kolumny (za znakiem $=$).
\item Podwójny backslash \keystrokebftt{\bs}\keystrokebftt{\bs} rozpoczyna nową linię.
}\end{itemize}
\end{frame}


%%%%%%%%%%%%%%%%%%%%%%%%%%%%%%%%%%%%%%%%%%%%%%%%%%%%%%%%%%%%%%%%%%%%%%%%%%%%%%%
%%%%%%%%%%%%%%%%%%%%%%%%%%%%%%%%%%%%%%%%%%%%%%%%%%%%%%%%%%%%%%%%%%%%%%%%%%%%%%%
%%%%%%%%%%%%%%%%%%%%%%%%%%%%%%%%%%%%%%%%%%%%%%%%%%%%%%%%%%%%%%%%%%%%%%%%%%%%%%%
\begin{frame}[fragile]{Skład tekstu Ćwiczenie 2}

\begin{block}{Wpisz poniższy tekst w \LaTeX u:}
Niech $X_1, X_2, \ldots, X_n$ będzie ciągiem niezależnych zmiennych losowych o
jednakowych rozkładach z parametrami $\operatorname{E}[X_i] = \mu$ i
$\operatorname{Var}[X_i] = \sigma^2 < \infty$, oraz niech
\begin{equation*}
S_n = \frac{1}{n}\sum_{i=1}^{n} X_i
\end{equation*}
oznaczyć ich średnią. Następnie, gdy $n$ zbliża się do nieskończoności, wtedy
zmienne losowe $\sqrt{n}(S_n - \mu)$ zbiegają do rozkładu normalnego $N(0, \sigma^2)$.
\end{block}
\vskip 2ex
\begin{center}
\fbox{\href{\wlnewdoc{podstawy-cwiczenie-2.tex}}{%
Kliknij, aby otworzyć to ćwiczenie w \wllogo{}}}
\end{center}
\begin{itemize}
\item Wskazówka: komenda dla $\infty$ to \cmdbs{infty}.
\item Kiedy już spróbujesz,
\fbox{\href{\wlnewdoc{podstawy-cwiczenie-2-rozwiazanie.tex}}{%
kliknij, aby zobaczyć rozwiązanie}}.
\end{itemize}
\end{frame}

%%%%%%%%%%%%%%%%%%%%%%%%%%%%%%%%%%%%%%%%%%%%%%%%%%%%%%%%%%%%%%%%%%%%%%%%%%%%%%%
%%%%%%%%%%%%%%%%%%%%%%%%%%%%%%%%%%%%%%%%%%%%%%%%%%%%%%%%%%%%%%%%%%%%%%%%%%%%%%%
%%%%%%%%%%%%%%%%%%%%%%%%%%%%%%%%%%%%%%%%%%%%%%%%%%%%%%%%%%%%%%%%%%%%%%%%%%%%%%%
\begin{frame}{End of Part 1}
\begin{itemize}
\item Congrats! You've already learned how to \ldots
\begin{itemize}
\item Typeset text in \LaTeX.
\item Use lots of different commands.
\item Handle errors when they arise.
\item Typeset some beautiful mathematics.
\item Use several different environments.
\item Load packages.
\end{itemize}
\item That's amazing!
\item In Part 2, we'll see how to use \LaTeX{} to write structured documents
with sections, cross references, figures, tables and bibliographies. See you
then!
\end{itemize}
\end{frame}

\end{document}
