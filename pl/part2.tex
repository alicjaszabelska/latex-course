\documentclass{beamer}

\input{preamble.tex}

\subtitle{Część 2: Ustrukturyzowane dokumenty \& więcej}

\begin{document}

%%%%%%%%%%%%%%%%%%%%%%%%%%%%%%%%%%%%%%%%%%%%%%%%%%%%%%%%%%%%%%%%%%%%%%%%%%%%%%%
%%%%%%%%%%%%%%%%%%%%%%%%%%%%%%%%%%%%%%%%%%%%%%%%%%%%%%%%%%%%%%%%%%%%%%%%%%%%%%%
%%%%%%%%%%%%%%%%%%%%%%%%%%%%%%%%%%%%%%%%%%%%%%%%%%%%%%%%%%%%%%%%%%%%%%%%%%%%%%%
\begin{frame}
\titlepage
\end{frame}

%%%%%%%%%%%%%%%%%%%%%%%%%%%%%%%%%%%%%%%%%%%%%%%%%%%%%%%%%%%%%%%%%%%%%%%%%%%%%%%
%%%%%%%%%%%%%%%%%%%%%%%%%%%%%%%%%%%%%%%%%%%%%%%%%%%%%%%%%%%%%%%%%%%%%%%%%%%%%%%
%%%%%%%%%%%%%%%%%%%%%%%%%%%%%%%%%%%%%%%%%%%%%%%%%%%%%%%%%%%%%%%%%%%%%%%%%%%%%%%
\section{Ustrukturyzowane dokumenty}

%%%%%%%%%%%%%%%%%%%%%%%%%%%%%%%%%%%%%%%%%%%%%%%%%%%%%%%%%%%%%%%%%%%%%%%%%%%%%%%
%%%%%%%%%%%%%%%%%%%%%%%%%%%%%%%%%%%%%%%%%%%%%%%%%%%%%%%%%%%%%%%%%%%%%%%%%%%%%%%
%%%%%%%%%%%%%%%%%%%%%%%%%%%%%%%%%%%%%%%%%%%%%%%%%%%%%%%%%%%%%%%%%%%%%%%%%%%%%%%
\begin{frame}{Spis treści}
\begin{multicols}{2}
\tableofcontents[currentsection]
\end{multicols}
\end{frame}

%%%%%%%%%%%%%%%%%%%%%%%%%%%%%%%%%%%%%%%%%%%%%%%%%%%%%%%%%%%%%%%%%%%%%%%%%%%%%%%
%%%%%%%%%%%%%%%%%%%%%%%%%%%%%%%%%%%%%%%%%%%%%%%%%%%%%%%%%%%%%%%%%%%%%%%%%%%%%%%
%%%%%%%%%%%%%%%%%%%%%%%%%%%%%%%%%%%%%%%%%%%%%%%%%%%%%%%%%%%%%%%%%%%%%%%%%%%%%%%
\begin{frame}{\insertsection}
\begin{itemize}
\item W Części 1 dowiedzieliśmy się o poleceniach i środowiskach do składu 
tekstu i matematyki.
\item Teraz dowiemy się o komendach i środowiskach służących do strukturyzacji 
dokumentów.
\item Możesz wypróbować nowe komendy w Overleaf:
\end{itemize}
\vskip 2em
\begin{center}
\fbox{\href{\wlnewdoc{podstawy.tex}}{%
Naciśnij tutaj, aby otworzyć przykładowy dokument w \wllogo{}}}
\\[1ex]\scriptsize{}
Dla najlepszych resultatów, proszę użyj \href{http://www.google.com/chrome}{Google Chrome} lub aktualnej wersji \href{http://www.mozilla.org/en-US/firefox/new/}{FireFox}.
\end{center}
\vskip 2ex
\begin{itemize}
\item Zaczynajmy!
\end{itemize}
\end{frame}

%%%%%%%%%%%%%%%%%%%%%%%%%%%%%%%%%%%%%%%%%%%%%%%%%%%%%%%%%%%%%%%%%%%%%%%%%%%%%%%
%%%%%%%%%%%%%%%%%%%%%%%%%%%%%%%%%%%%%%%%%%%%%%%%%%%%%%%%%%%%%%%%%%%%%%%%%%%%%%%
%%%%%%%%%%%%%%%%%%%%%%%%%%%%%%%%%%%%%%%%%%%%%%%%%%%%%%%%%%%%%%%%%%%%%%%%%%%%%%%
\subsection{Tytuł i streszczenie}
\begin{frame}[fragile]{\insertsubsection}
\begin{itemize}{\small
\item Podaj \LaTeX{}'owi nazwy \cmdbs{title} i \cmdbs{author} w preambule.
\item Następnie użyj \cmdbs{maketitle} w dokumencie, aby faktycznie utworzyć tytuł.
\item Użyj środowiska \bftt{abstract}, aby utworzyć streszczenie.
}\end{itemize}
\begin{minipage}{0.55\linewidth}
\inputminted[fontsize=\scriptsize,frame=single,resetmargins]{latex}%
  {struktura-tytul.tex}
\end{minipage}
\begin{minipage}{0.35\linewidth}
\includegraphics[width=\textwidth,clip,trim=2.2in 7in 2.2in 2in]{struktura-tytul.pdf}
\end{minipage}
\end{frame}

%%%%%%%%%%%%%%%%%%%%%%%%%%%%%%%%%%%%%%%%%%%%%%%%%%%%%%%%%%%%%%%%%%%%%%%%%%%%%%%
%%%%%%%%%%%%%%%%%%%%%%%%%%%%%%%%%%%%%%%%%%%%%%%%%%%%%%%%%%%%%%%%%%%%%%%%%%%%%%%
%%%%%%%%%%%%%%%%%%%%%%%%%%%%%%%%%%%%%%%%%%%%%%%%%%%%%%%%%%%%%%%%%%%%%%%%%%%%%%%
\subsection{Sekcje}
\begin{frame}{\insertsubsection}
\begin{itemize}{\small
\item Po prostu użyj \cmdbs{section} i \cmdbs{subsection}.
\item Czy wiesz, co robią \cmdbs{section*} i \cmdbs{subsection*}?
}\end{itemize}
\begin{minipage}{0.55\linewidth}
\inputminted[fontsize=\scriptsize,frame=single,resetmargins]{latex}%
  {struktura-sekcje.tex}
\end{minipage}
\begin{minipage}{0.35\linewidth}
\includegraphics[width=\textwidth,clip,trim=1.5in 6in 4in 1in]{struktura-sekcje.pdf}
\end{minipage}
\end{frame}

%%%%%%%%%%%%%%%%%%%%%%%%%%%%%%%%%%%%%%%%%%%%%%%%%%%%%%%%%%%%%%%%%%%%%%%%%%%%%%%
%%%%%%%%%%%%%%%%%%%%%%%%%%%%%%%%%%%%%%%%%%%%%%%%%%%%%%%%%%%%%%%%%%%%%%%%%%%%%%%
%%%%%%%%%%%%%%%%%%%%%%%%%%%%%%%%%%%%%%%%%%%%%%%%%%%%%%%%%%%%%%%%%%%%%%%%%%%%%%%
\subsection{Etykiety i odwołania}
\begin{frame}[fragile]{\insertsubsection}
\begin{itemize}{\small
\item Użyj \cmdbs{label} i \cmdbs{ref} do automatycznego numerowania.
\item Pakiet \bftt{amsmath} dostarcza \cmdbs{eqref} do odwoływania się do równań.
}\end{itemize}
\begin{minipage}{0.55\linewidth}
\inputminted[fontsize=\scriptsize,frame=single,resetmargins]{latex}%
  {struktura-odwolania.tex}
\end{minipage}
\begin{minipage}{0.35\linewidth}
\includegraphics[width=\textwidth,clip,trim=1.8in 6in 1.6in 1in]{struktura-odwolania.pdf}
\end{minipage}
\end{frame}

%%%%%%%%%%%%%%%%%%%%%%%%%%%%%%%%%%%%%%%%%%%%%%%%%%%%%%%%%%%%%%%%%%%%%%%%%%%%%%%
%%%%%%%%%%%%%%%%%%%%%%%%%%%%%%%%%%%%%%%%%%%%%%%%%%%%%%%%%%%%%%%%%%%%%%%%%%%%%%%
%%%%%%%%%%%%%%%%%%%%%%%%%%%%%%%%%%%%%%%%%%%%%%%%%%%%%%%%%%%%%%%%%%%%%%%%%%%%%%%
\subsection{Ćwiczenie}
\begin{frame}[fragile]{Ustrukturyzowane dokumenty Ćwiczenie}

\begin{block}{Złóż ten krótki artykuł w \LaTeX u:
\footnote{Z \url{http://pdos.csail.mit.edu/scigen/}, losowy
generator artykułu.}}
\begin{center}
\fbox{\href{\fileuri/struktura-cwiczenie-rozwiazanie.pdf}{%
Kliknij, aby otworzyć dokument}}
\end{center}
Spraw, aby twój papier wyglądał jak ten. Użyj \cmdbs{ref} i \cmdbs{eqref}, aby uniknąć jawnego wpisywania numerów sekcji i równań w tekście.
\end{block}
\vskip 2ex
\begin{center}
\fbox{\href{\wlnewdoc{struktura-cwiczenie.tex}}{%
Kliknij, aby otworzyć to ćwiczenie w \wllogo{}}}
\end{center}

\begin{itemize}
\item Kiedy już spróbujesz,
\fbox{\href{\wlnewdoc{struktura-cwiczenie-rozwiazanie.tex}}{%
kliknij tutaj, aby zobaczyć rozwiązanie}}.
\end{itemize}
\end{frame}

%%%%%%%%%%%%%%%%%%%%%%%%%%%%%%%%%%%%%%%%%%%%%%%%%%%%%%%%%%%%%%%%%%%%%%%%%%%%%%%
%%%%%%%%%%%%%%%%%%%%%%%%%%%%%%%%%%%%%%%%%%%%%%%%%%%%%%%%%%%%%%%%%%%%%%%%%%%%%%%
%%%%%%%%%%%%%%%%%%%%%%%%%%%%%%%%%%%%%%%%%%%%%%%%%%%%%%%%%%%%%%%%%%%%%%%%%%%%%%%
\section{Figury i tabele}

%%%%%%%%%%%%%%%%%%%%%%%%%%%%%%%%%%%%%%%%%%%%%%%%%%%%%%%%%%%%%%%%%%%%%%%%%%%%%%%
%%%%%%%%%%%%%%%%%%%%%%%%%%%%%%%%%%%%%%%%%%%%%%%%%%%%%%%%%%%%%%%%%%%%%%%%%%%%%%%
%%%%%%%%%%%%%%%%%%%%%%%%%%%%%%%%%%%%%%%%%%%%%%%%%%%%%%%%%%%%%%%%%%%%%%%%%%%%%%%
\begin{frame}{Spi treści}
\begin{multicols}{2}
\tableofcontents[currentsection]
\end{multicols}
\end{frame}

%%%%%%%%%%%%%%%%%%%%%%%%%%%%%%%%%%%%%%%%%%%%%%%%%%%%%%%%%%%%%%%%%%%%%%%%%%%%%%%
%%%%%%%%%%%%%%%%%%%%%%%%%%%%%%%%%%%%%%%%%%%%%%%%%%%%%%%%%%%%%%%%%%%%%%%%%%%%%%%
%%%%%%%%%%%%%%%%%%%%%%%%%%%%%%%%%%%%%%%%%%%%%%%%%%%%%%%%%%%%%%%%%%%%%%%%%%%%%%%
\subsection{Grafika}
\begin{frame}[fragile]{\insertsubsection}
\begin{itemize}
\item Wymaga pakietu \bftt{graphicx}, który udostępnia komenda \cmdbs{includegraphics}.
\item Obsługiwane formaty graficzne to JPEG, PNG i PDF (zwykle).
\end{itemize}
\begin{exampletwouptiny}
\includegraphics[
  width=0.5\textwidth]{gerbil}

\includegraphics[
  width=0.3\textwidth,
  angle=270]{gerbil}
\end{exampletwouptiny}

\tiny{Licencja obrazu: \href{https://pixabay.com/en/animal-apple-attractive-beautiful-1239390/}{CC0}}
\end{frame}

%%%%%%%%%%%%%%%%%%%%%%%%%%%%%%%%%%%%%%%%%%%%%%%%%%%%%%%%%%%%%%%%%%%%%%%%%%%%%%%
%%%%%%%%%%%%%%%%%%%%%%%%%%%%%%%%%%%%%%%%%%%%%%%%%%%%%%%%%%%%%%%%%%%%%%%%%%%%%%%
%%%%%%%%%%%%%%%%%%%%%%%%%%%%%%%%%%%%%%%%%%%%%%%%%%%%%%%%%%%%%%%%%%%%%%%%%%%%%%%
\begin{frame}[fragile]{Interludium: opcjonalne argumenty}
\begin{itemize}
\item Używamy nawiasów kwadratowych \keystrokebftt{[} \keystrokebftt{]} jako
opcjonalnych argumentów zamiast nawiasów klamrowych \keystrokebftt{\{} \keystrokebftt{\}}.
\item \cmdbs{includegraphics} akceptuje opcjonalne argumenty, które pozwalają 
na przekształcenie obrazu, gdy jest on zawarty. Na przykład
\bftt{width=0.3\cmdbs{textwidth}} powoduje, że obraz zajmuje 30\% szerokości 
otaczającego tekstu (\cmdbs{textwidth}).
\item \cmdbs{documentclass} akceptuje również argumenty opcjonalne. Przykład:
\mint{latex}|\documentclass[12pt,twocolumn]{article}|
\vskip 3ex
powiększa tekst (12pt) i umieszcza go w dwóch kolumnach.
\item Skąd się o nich dowiesz? Zobacz slajdy na końcu tej prezentacji, aby 
uzyskać linki do dodatkowych informacji.
\end{itemize}
\end{frame}

%%%%%%%%%%%%%%%%%%%%%%%%%%%%%%%%%%%%%%%%%%%%%%%%%%%%%%%%%%%%%%%%%%%%%%%%%%%%%%%
%%%%%%%%%%%%%%%%%%%%%%%%%%%%%%%%%%%%%%%%%%%%%%%%%%%%%%%%%%%%%%%%%%%%%%%%%%%%%%%
%%%%%%%%%%%%%%%%%%%%%%%%%%%%%%%%%%%%%%%%%%%%%%%%%%%%%%%%%%%%%%%%%%%%%%%%%%%%%%%
\subsection[fragile]{"Pływanie" tekstu}
\begin{frame}{\insertsubsection}
\begin{itemize}
\item Pozwól, aby \LaTeX{} zdecydował, gdzie trafi figura (może ``pływać'').
\item Możesz również nadać figurze podpis, do którego można się odwoływać za 
pomocą \cmdbs{ref}.
\end{itemize}
\begin{minipage}{0.55\linewidth}
\inputminted[fontsize=\scriptsize,frame=single,resetmargins]{latex}%
  {media-grafika.tex}
\end{minipage}
\begin{minipage}{0.35\linewidth}
\includegraphics[width=\textwidth,clip,trim=2in 5in 3in 1in]{media-grafika.pdf}
\end{minipage}

\tiny{Licencja obrazu: \href{https://pixabay.com/en/animal-apple-attractive-beautiful-1239390/}{CC0}}
\end{frame}

%%%%%%%%%%%%%%%%%%%%%%%%%%%%%%%%%%%%%%%%%%%%%%%%%%%%%%%%%%%%%%%%%%%%%%%%%%%%%%%
%%%%%%%%%%%%%%%%%%%%%%%%%%%%%%%%%%%%%%%%%%%%%%%%%%%%%%%%%%%%%%%%%%%%%%%%%%%%%%%
%%%%%%%%%%%%%%%%%%%%%%%%%%%%%%%%%%%%%%%%%%%%%%%%%%%%%%%%%%%%%%%%%%%%%%%%%%%%%%%
\subsection{Tabele}
\begin{frame}[fragile]{\insertsubsection}
\begin{itemize}
\item Przyzwyczajenie się do tabel w \LaTeX{} wymaga trochę czasu.
\item Użyj środowiska \bftt{tabular} z pakietu \bftt{tabularx}.
\item Argument określa wyrównanie kolumn --- \textbf{l}eft, \textbf{r}ight, \textbf{r}ight.
\begin{exampletwouptiny}
\begin{tabular}{lrr}
Pozycja   & Ilość & Jedn. \$ \\
Widżet & 1   & 199.99  \\
Gadżet & 2   & 399.99  \\
Kabel  & 3   & 19.99   \\
\end{tabular}
\end{exampletwouptiny}
\item Określa również linie pionowe; użyj \cmdbs{hline} dla linii poziomych.
\begin{exampletwouptiny}
\begin{tabular}{|l|r|r|} \hline
Pozycja   & Ilość & Jedn. \$ \\\hline
Widżet & 1   & 199.99  \\
Gadżet & 2   & 399.99  \\
Kabel  & 3   & 19.99   \\\hline
\end{tabular}
\end{exampletwouptiny}
\item Użyj klawisza et \keystrokebftt{\&}, aby oddzielić kolumny, a podwójnego backslashu \keystrokebftt{\bs}\keystrokebftt{\bs}, aby rozpocząć nowy wiersz (jak w środowisku \bftt{align*}, które widzieliśmy w części 1).
\end{itemize}
\end{frame}

%%%%%%%%%%%%%%%%%%%%%%%%%%%%%%%%%%%%%%%%%%%%%%%%%%%%%%%%%%%%%%%%%%%%%%%%%%%%%%%
%%%%%%%%%%%%%%%%%%%%%%%%%%%%%%%%%%%%%%%%%%%%%%%%%%%%%%%%%%%%%%%%%%%%%%%%%%%%%%%
%%%%%%%%%%%%%%%%%%%%%%%%%%%%%%%%%%%%%%%%%%%%%%%%%%%%%%%%%%%%%%%%%%%%%%%%%%%%%%%
\addtocontents{toc}{\newpage}
\section{Bibliografie}

%%%%%%%%%%%%%%%%%%%%%%%%%%%%%%%%%%%%%%%%%%%%%%%%%%%%%%%%%%%%%%%%%%%%%%%%%%%%%%%
%%%%%%%%%%%%%%%%%%%%%%%%%%%%%%%%%%%%%%%%%%%%%%%%%%%%%%%%%%%%%%%%%%%%%%%%%%%%%%%
%%%%%%%%%%%%%%%%%%%%%%%%%%%%%%%%%%%%%%%%%%%%%%%%%%%%%%%%%%%%%%%%%%%%%%%%%%%%%%%
\begin{frame}{Spis treści}
\begin{multicols}{2}
\tableofcontents[currentsection]
\end{multicols}
\end{frame}

%%%%%%%%%%%%%%%%%%%%%%%%%%%%%%%%%%%%%%%%%%%%%%%%%%%%%%%%%%%%%%%%%%%%%%%%%%%%%%%
%%%%%%%%%%%%%%%%%%%%%%%%%%%%%%%%%%%%%%%%%%%%%%%%%%%%%%%%%%%%%%%%%%%%%%%%%%%%%%%
%%%%%%%%%%%%%%%%%%%%%%%%%%%%%%%%%%%%%%%%%%%%%%%%%%%%%%%%%%%%%%%%%%%%%%%%%%%%%%%
\subsection{bib\TeX}
\begin{frame}[fragile]{\insertsubsection{} 1}
\begin{itemize}
\item Umieść swoje referencje w pliku \bftt{.bib} w formacie bazy danych `bibtex':
\inputminted[fontsize=\scriptsize,frame=single]{latex}{bib-przyklad.bib}
\item Większość menedżerów referencji może eksportować do formatu bitex.
\end{itemize}
\end{frame}

%%%%%%%%%%%%%%%%%%%%%%%%%%%%%%%%%%%%%%%%%%%%%%%%%%%%%%%%%%%%%%%%%%%%%%%%%%%%%%%
%%%%%%%%%%%%%%%%%%%%%%%%%%%%%%%%%%%%%%%%%%%%%%%%%%%%%%%%%%%%%%%%%%%%%%%%%%%%%%%
%%%%%%%%%%%%%%%%%%%%%%%%%%%%%%%%%%%%%%%%%%%%%%%%%%%%%%%%%%%%%%%%%%%%%%%%%%%%%%%
\begin{frame}[fragile]{\insertsubsection{} 2}
\begin{itemize}
\item Każdy wpis w pliku \bftt{.bib} ma \emph{klucz}, którego można
użyć, aby odnieść się do niego w dokumencie. Na przykład \bftt{Jacobson1999Towards} jest kluczem do tego artykułu:
\begin{minted}[fontsize=\small,frame=single]{latex}
@Article{Jacobson1999Towards,
  author = {Van Jacobson},
  ...
}
\end{minted}
\item Dobrym pomysłem jest użycie klucza opartego na nazwisku, roku i tytule.
\item \LaTeX{} może automatycznie formatować cytaty w tekście i 
generować listę odniesień; zna większość standardowych stylów i 
możesz zaprojektować własny.
\end{itemize}
\end{frame}

%%%%%%%%%%%%%%%%%%%%%%%%%%%%%%%%%%%%%%%%%%%%%%%%%%%%%%%%%%%%%%%%%%%%%%%%%%%%%%%
%%%%%%%%%%%%%%%%%%%%%%%%%%%%%%%%%%%%%%%%%%%%%%%%%%%%%%%%%%%%%%%%%%%%%%%%%%%%%%%
%%%%%%%%%%%%%%%%%%%%%%%%%%%%%%%%%%%%%%%%%%%%%%%%%%%%%%%%%%%%%%%%%%%%%%%%%%%%%%%
\begin{frame}[fragile]{\insertsubsection{} 3}
\begin{itemize}
\item Użyj pakietu \bftt{natbib}\footnote{Jest nowy pakiet z większą liczbą funkcji o nazwie \bftt{biblatex}, ale większość szablonów artykułów nadal używa \bftt{natbib}.} z \cmdbs{citet} i \cmdbs{citep}.
\item Reference \cmdbs{bibliography} at the end, and specify a \cmdbs{bibliographystyle}.
\end{itemize}
\begin{minipage}{0.55\linewidth}
\inputminted[fontsize=\tiny,frame=single,resetmargins]{latex}%
  {bib-przyklad.tex}
\end{minipage}
\begin{minipage}{0.35\linewidth}
\includegraphics[width=\textwidth,clip,trim=1.8in 5in 1.8in 1in]{bib-przyklad.pdf}
\end{minipage}
\end{frame}

%%%%%%%%%%%%%%%%%%%%%%%%%%%%%%%%%%%%%%%%%%%%%%%%%%%%%%%%%%%%%%%%%%%%%%%%%%%%%%%
%%%%%%%%%%%%%%%%%%%%%%%%%%%%%%%%%%%%%%%%%%%%%%%%%%%%%%%%%%%%%%%%%%%%%%%%%%%%%%%
%%%%%%%%%%%%%%%%%%%%%%%%%%%%%%%%%%%%%%%%%%%%%%%%%%%%%%%%%%%%%%%%%%%%%%%%%%%%%%%
\subsection{Ćwiczenie}
\begin{frame}[fragile]{Ćwiczenie: Łączenie wszystkiego w całość}

Dodaj zdjęcie i bibliografię do pracy z poprzedniego ćwiczenia.

\begin{enumerate}
\item Pobierz te przykładowe pliki na swój komputer.

\begin{center}
\fbox{\href{\fileuri/gerbil.jpg?dl=1}{Kliknij, aby pobrać przykładowy obraz}}

\fbox{\href{\fileuri/bib-cwiczenie.bib?dl=1}{Kliknij, aby pobrać przykładowy plik bib}}
\end{center}

\item Prześlij je na Overleaf (użyj menu projektu).

\end{enumerate}
\end{frame}

%%%%%%%%%%%%%%%%%%%%%%%%%%%%%%%%%%%%%%%%%%%%%%%%%%%%%%%%%%%%%%%%%%%%%%%%%%%%%%%
%%%%%%%%%%%%%%%%%%%%%%%%%%%%%%%%%%%%%%%%%%%%%%%%%%%%%%%%%%%%%%%%%%%%%%%%%%%%%%%
%%%%%%%%%%%%%%%%%%%%%%%%%%%%%%%%%%%%%%%%%%%%%%%%%%%%%%%%%%%%%%%%%%%%%%%%%%%%%%%
\section{Co dalej?}

%%%%%%%%%%%%%%%%%%%%%%%%%%%%%%%%%%%%%%%%%%%%%%%%%%%%%%%%%%%%%%%%%%%%%%%%%%%%%%%
%%%%%%%%%%%%%%%%%%%%%%%%%%%%%%%%%%%%%%%%%%%%%%%%%%%%%%%%%%%%%%%%%%%%%%%%%%%%%%%
%%%%%%%%%%%%%%%%%%%%%%%%%%%%%%%%%%%%%%%%%%%%%%%%%%%%%%%%%%%%%%%%%%%%%%%%%%%%%%%
\begin{frame}{Spis treści}
\begin{multicols}{2}
\tableofcontents[currentsection]
\end{multicols}
\end{frame}

%%%%%%%%%%%%%%%%%%%%%%%%%%%%%%%%%%%%%%%%%%%%%%%%%%%%%%%%%%%%%%%%%%%%%%%%%%%%%%%
%%%%%%%%%%%%%%%%%%%%%%%%%%%%%%%%%%%%%%%%%%%%%%%%%%%%%%%%%%%%%%%%%%%%%%%%%%%%%%%
%%%%%%%%%%%%%%%%%%%%%%%%%%%%%%%%%%%%%%%%%%%%%%%%%%%%%%%%%%%%%%%%%%%%%%%%%%%%%%%
\subsection{Więcej schludnych rzeczy}
\begin{frame}[fragile]{\insertsubsection}
\begin{itemize}
\item Dodaj komendę \cmdbs{tableofcontents}, aby wygenerować spis treści z zawartości komend \cmdbs{section}.

\item Zmień \cmdbs{documentclass} na
\mint{latex}!\documentclass{scrartcl}!
lub
\mint{latex}!\documentclass[12pt]{IEEEtran}!

\item Zdefiniuj własne polecenie dla skomplikowanego równania:
\begin{exampletwouptiny}
\newcommand{\rperf}{%
  \rho_{\text{perf}}}
$$
\rperf = {\bf c}'{\bf X} + \varepsilon
$$
\end{exampletwouptiny}
\end{itemize}
\end{frame}

%%%%%%%%%%%%%%%%%%%%%%%%%%%%%%%%%%%%%%%%%%%%%%%%%%%%%%%%%%%%%%%%%%%%%%%%%%%%%%%
%%%%%%%%%%%%%%%%%%%%%%%%%%%%%%%%%%%%%%%%%%%%%%%%%%%%%%%%%%%%%%%%%%%%%%%%%%%%%%%
%%%%%%%%%%%%%%%%%%%%%%%%%%%%%%%%%%%%%%%%%%%%%%%%%%%%%%%%%%%%%%%%%%%%%%%%%%%%%%%
\subsection{Więcej schludnych pakietów}
\begin{frame}{\insertsubsection}
\begin{itemize}
\item \bftt{beamer}: do prezentacji (takich jak ta!)
\item \bftt{todonotes}: zarządzanie komentarzami i TODO
\item \bftt{tikz}: tworzyć niesamowite grafiki
\item \bftt{pgfplots}: tworzyć wykresy w \LaTeX
\item \bftt{listings}: drukarka kodu źródłowego dla \LaTeX
\item \bftt{spreadtab}: tworzyć arkusze kalkulacyjne w \LaTeX
\item \bftt{gchords}, \bftt{guitar}: akordy gitarowe i tabulatury
\item \bftt{cwpuzzle}: Krzyżówki
\end{itemize}
Zobacz \url{https://www.overleaf.com/latex/examples} oraz \url{http://texample.net}
dla przykładów (większości) tych pakietów.
\end{frame}

%%%%%%%%%%%%%%%%%%%%%%%%%%%%%%%%%%%%%%%%%%%%%%%%%%%%%%%%%%%%%%%%%%%%%%%%%%%%%%%
%%%%%%%%%%%%%%%%%%%%%%%%%%%%%%%%%%%%%%%%%%%%%%%%%%%%%%%%%%%%%%%%%%%%%%%%%%%%%%%
%%%%%%%%%%%%%%%%%%%%%%%%%%%%%%%%%%%%%%%%%%%%%%%%%%%%%%%%%%%%%%%%%%%%%%%%%%%%%%%
\subsection{Instalowanie \LaTeX{} a}
\begin{frame}{\insertsubsection}
\begin{itemize}
\item Aby uruchomić \LaTeX{} na własnym komputerze, będziesz chciał użyć dystrybucji \LaTeX{}. Dystrybucja zawiera program \bftt{latex} i (zazwyczaj) kilka tysięcy pakietów.
\begin{itemize}
\item Na Windows'ie: \href{http://miktex.org/}{Mik\TeX} or \href{http://tug.org/texlive/}{\TeX Live}
\item Na Linux'ie: \href{http://tug.org/texlive/}{\TeX Live}
\item Na Mac'u: \href{http://tug.org/mactex/}{Mac\TeX}
\end{itemize}
\item Będziesz także potrzebować edytora tekstu z obsługą \LaTeX{}. Zobacz adres URL{http://en.wikipedia.org/wiki/Comparison\_of\_TeX\_editors}, aby zobaczyć listę (wielu) opcji.
\item Musisz także dowiedzieć się więcej o tym, jak działa \bftt{latex} i powiązane z nim narzędzia --- zobacz zasoby na następnym slajdzie.
\end{itemize}
\end{frame}

%%%%%%%%%%%%%%%%%%%%%%%%%%%%%%%%%%%%%%%%%%%%%%%%%%%%%%%%%%%%%%%%%%%%%%%%%%%%%%%
%%%%%%%%%%%%%%%%%%%%%%%%%%%%%%%%%%%%%%%%%%%%%%%%%%%%%%%%%%%%%%%%%%%%%%%%%%%%%%%
%%%%%%%%%%%%%%%%%%%%%%%%%%%%%%%%%%%%%%%%%%%%%%%%%%%%%%%%%%%%%%%%%%%%%%%%%%%%%%%
\subsection{Zasoby online}
\begin{frame}{\insertsubsection}
\begin{itemize}
\item \href{https://www.overleaf.com/learn}{Wiki platformy Overleaf} ---
zawiera te slajdy (w wersji angielskiej), więcej samouczków i materiałów referencyjnych
\item \href{http://en.wikibooks.org/wiki/LaTeX}{Wikibook \LaTeX{} a} ---
doskonałe samouczki i materiały referencyjne.
\item \href{http://tex.stackexchange.com/}{\TeX{} Stack Exchange} --- forum do zadawania pytań, gdzie uzyskuje się doskonałe odpowiedzi niezwykle szybko
\item \href{http://www.latex-community.org/}{\LaTeX{} Community} --- duże forum internetowe
\item \href{http://ctan.org/}{Comprehensive \TeX{} Archive Network (CTAN)} ---
ponad cztery tysiące paczek plus dokumentacja
\item Google zwykle prowadzi do jednego z powyższych.
\end{itemize}
\end{frame}

%%%%%%%%%%%%%%%%%%%%%%%%%%%%%%%%%%%%%%%%%%%%%%%%%%%%%%%%%%%%%%%%%%%%%%%%%%%%%%%
%%%%%%%%%%%%%%%%%%%%%%%%%%%%%%%%%%%%%%%%%%%%%%%%%%%%%%%%%%%%%%%%%%%%%%%%%%%%%%%
%%%%%%%%%%%%%%%%%%%%%%%%%%%%%%%%%%%%%%%%%%%%%%%%%%%%%%%%%%%%%%%%%%%%%%%%%%%%%%%
\begin{frame}
\begin{center}
Dziękujemy i życzymy udanej pracy z \TeX{}-em!
\end{center}
\end{frame}

\end{document}

% -- latex understands words, sentences and paragraphs

Words are separated by one or more spaces.  Paragraphs are separated by
one or more blank lines.  The output is not affected by adding extra
spaces or extra blank lines to the input file.

Double quotes are typed like this: ``quoted text''.
Single quotes are typed like this: `single-quoted text'.

Emphasized text is typed like this: \emph{this is emphasized}.
Bold       text is typed like this: \textbf{this is bold}.

-- Adding structure to your document

\section{Hello}

\subsection{World}

\subsection{Foo}

\subsubsection*{Stuff} % star form

\subsubsection*{Results}

-- Labels and cross-references

\label{sec:intro}
\label{sec:method}
\ref{sec:method}

--> maybe introduce the prettyref package here.

-- Mathematics

Inline mathematics: $x + y < 7$.

'Displayed' mathematics:
\begin{equation}
\end{equation}

\begin{equation*}
\end{equation*}

\begin{align}
\end{align}

-- Figures

- Need the graphicx package.

- here we can start introducing options

\includegraphics[width=\textwidth]{}

- where do you find out about these options? --> link to the Wikibook

-- Floating Figures

\begin{figure}
\includegraphics{...}
\caption{\label{}Here is a caption.}
\end{figure}

-- Tables

- not the nicest part of LaTeX

\usepackage{tabularx}

\begin{tabular}{llr}
Item & Quantity & Price (\$) & Amount
Widget & 1 &
\end{tabular}

Bonus points: check out the fp package and the spreadtab package.

-- Document Classes

a .cls file

article

some journal templates come with one

-- Bibliographies



-- For Typesetting Geeks

- dashes: -, --, ---

- ellipsis.

- controlling spaces: ~, \ , \,, \@

- spacing after periods (et al., etc.)

- Nested quotation marks: ``\,`
\vskip 2ex
\item Use the \emph{star form} to display an equation without a number.
\begin{exampletwouptiny}
\begin{equation*}
F(x) = \int_{a}^{x}{f(t) dt}
\end{equation*}
\end{exampletwouptiny}

\begin{itemize}
\item \bftt{equation} and \bftt{equation*} are called \emph{environments}.
\begin{itemize}
  \item The \cmdbs{begin} and \cmdbs{end} commands define the environment.
  \item The \cmd{\$} also starts and ends an environment.
  \item Some commands are defined only within certain environments.
  \item Some commands behave differently in different environments.
\end{itemize}
\end{itemize}
\end{block}
\begin{center}
\fbox{\href{http://ctan.org/}{The Comprehensive \TeX Archive Network (CTAN)}}
\end{center}

%%%%%%%%%%%%%%%%%%%%%%%%%%%%%%%%%%%%%%%%%%%%%%%%%%%%%%%%%%%%%%%%%%%%%%%%%%%%%%%
%%%%%%%%%%%%%%%%%%%%%%%%%%%%%%%%%%%%%%%%%%%%%%%%%%%%%%%%%%%%%%%%%%%%%%%%%%%%%%%
%%%%%%%%%%%%%%%%%%%%%%%%%%%%%%%%%%%%%%%%%%%%%%%%%%%%%%%%%%%%%%%%%%%%%%%%%%%%%%%
\subsection{Typography tweaks}
\begin{frame}{\insertsubsection}
\begin{tabular}{lll}
& character name & used mainly for \ldots \\\hline
\bftt{\bs} & backslash                 & commands, tables \\
\bftt{\{}  & open brace                & commands \\
\bftt{\}}  & close brace               & commands \\
\bftt{\%}  & percent sign              & comments \\
\bftt{\#}  & hash (pound / sharp) sign & custom commands \\
\bftt{\$}  & dollar sign               & equations \\
\bftt{\_}  & underscore                & equations (subscripts) \\
\bftt{\^}  & caret                     & equations (superscripts) \\
\bftt{\&}  & ampersand                 & tables \\
\bftt{\~}  & tilde                     & spacing \\
\end{tabular}
\end{frame}

%\item We've used several environments:
%\vskip 1ex
%{\scriptsize
%\begin{tabular}{ll}
%\cmdbs{begin}\bftt{\{document\}}\ldots\cmdbs{end}\bftt{\{document\}} &
%  document environment \\
%\cmdbs{begin}\bftt{\{itemize\}}\ldots\cmdbs{end}\bftt{\{itemize\}} &
%  itemized list environment \\
%\bftt{\$\ldots\$}     & \emph{in-text} math environment \\
%\bftt{\$\$\ldots\$\$} & \emph{displayed} math environment \\
%\cmdbs{begin}\bftt{\{equation\}}\ldots\cmdbs{end}\bftt{\{equation\}} &
%  displayed math environment w/ number
%\end{tabular}
%}
