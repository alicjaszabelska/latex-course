\documentclass{article}
\usepackage[polish]{babel}
\usepackage[utf8]{inputenc}
\usepackage[T1]{fontenc}

\begin{document}

% wskazówki:
% 1. możesz chcieć użyć pakietu amsmath; zwróć uwagę, że 
%    zostały już umieszczone 3 pakiety związane z polskimi znakami
% 2. komenda dla symbolu nieskończoności to \infty

Niech X1, X2, ..., Xn będzie ciągiem niezależnych
zmiennych losowych o jednakowych rozkładach z parametrami 
E[Xi] = mu i Var[Xi] = sigma kwadrat < nieskończoność, 
oraz niech

Sn = 1/n razy suma od i równego 1 do n z Xi

oznaczyć ich średnią. Następnie, gdy n zbliża się do 
nieskończoności, wtedy zmienne losowe pierwiastek z n(Sn - mu) 
zbiegają do rozkładu normalnego N(0, sigma kwadrat ).

\end{document}

